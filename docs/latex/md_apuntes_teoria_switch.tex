{\bfseries{Apunte Mejorado\+: Sentencia switch en C}}

\DoxyHorRuler{0}


{\bfseries{Introducción}} ~\newline
 La sentencia {\ttfamily switch} es una estructura de control que permite seleccionar una entre múltiples alternativas, basándose en el valor de una {\bfseries{expresión de control (selector)}}. Es ideal cuando las decisiones dependen de valores discretos (ej\+: menús, categorías, opciones).

\DoxyHorRuler{0}


{\bfseries{Sintaxis Básica}} ~\newline
 
\begin{DoxyCode}{0}
\DoxyCodeLine{\textcolor{keywordflow}{switch} (selector) \{}
\DoxyCodeLine{    \textcolor{keywordflow}{case} etiqueta1: }
\DoxyCodeLine{        sentencias1;}
\DoxyCodeLine{        \textcolor{keywordflow}{break};}
\DoxyCodeLine{    \textcolor{keywordflow}{case} etiqueta2: }
\DoxyCodeLine{        sentencias2;}
\DoxyCodeLine{        \textcolor{keywordflow}{break};}
\DoxyCodeLine{    \textcolor{keywordflow}{default}: }
\DoxyCodeLine{        sentencias\_default; \textcolor{comment}{// Opcional}}
\DoxyCodeLine{\}}

\end{DoxyCode}


\DoxyHorRuler{0}


{\bfseries{Reglas Importantes}} ~\newline

\begin{DoxyEnumerate}
\item El {\bfseries{selector}} debe ser de tipo {\ttfamily int} o {\ttfamily char} (no {\ttfamily float}, {\ttfamily double} o strings). ~\newline

\item Las {\bfseries{etiquetas}} deben ser constantes únicas (ej\+: {\ttfamily case 5\+:}, `case \textquotesingle{}A'\+:{\ttfamily ). \texorpdfstring{$<$}{<}br\texorpdfstring{$>$}{>}}
\item {\ttfamily El}default{\ttfamily es opcional, pero recomendado para manejar valores inesperados. \texorpdfstring{$<$}{<}br\texorpdfstring{$>$}{>}}
\item {\ttfamily La sentencia}break{\ttfamily termina la ejecución del}switch\`{}. Sin ella, se ejecutarán todos los casos siguientes ({\bfseries{fall-\/through}}). ~\newline

\end{DoxyEnumerate}

\DoxyHorRuler{0}


{\bfseries{IA\+: Tabla de Operadores Comunes}} ~\newline
 \tabulinesep=1mm
\begin{longtabu}spread 0pt [c]{*{3}{|X[-1]}|}
\hline
\PBS\centering \cellcolor{\tableheadbgcolor}\textbf{ Operador   }&\PBS\centering \cellcolor{\tableheadbgcolor}\textbf{ Uso   }&\PBS\centering \cellcolor{\tableheadbgcolor}\textbf{ Ejemplo    }\\\cline{1-3}
\endfirsthead
\hline
\endfoot
\hline
\PBS\centering \cellcolor{\tableheadbgcolor}\textbf{ Operador   }&\PBS\centering \cellcolor{\tableheadbgcolor}\textbf{ Uso   }&\PBS\centering \cellcolor{\tableheadbgcolor}\textbf{ Ejemplo    }\\\cline{1-3}
\endhead
Incremento   &{\ttfamily variable++}   &{\ttfamily i++;}    \\\cline{1-3}
Decremento   &{\ttfamily variable-\/-\/}   &{\ttfamily j-\/-\/;}    \\\cline{1-3}
Suma y asignación   &{\ttfamily variable +=}   &{\ttfamily k += 2;}    \\\cline{1-3}
Lógico AND   &{\ttfamily \&\&}   &{\ttfamily if (a \&\& b)}    \\\cline{1-3}
Lógico OR   &{\ttfamily \textbackslash{}$\vert$\textbackslash{}$\vert$}   &{\ttfamily if (a \textbackslash{}$\vert$\textbackslash{}$\vert$ b)}   \\\cline{1-3}
\end{longtabu}


\DoxyHorRuler{0}


{\bfseries{Ejemplo 1\+: Selección Simple}} ~\newline
 
\begin{DoxyCode}{0}
\DoxyCodeLine{\textcolor{keywordflow}{switch} (opcion) \{}
\DoxyCodeLine{    \textcolor{keywordflow}{case} 0: }
\DoxyCodeLine{        puts(\textcolor{stringliteral}{"{}Cero!"{}});}
\DoxyCodeLine{        \textcolor{keywordflow}{break};}
\DoxyCodeLine{    \textcolor{keywordflow}{case} 1: }
\DoxyCodeLine{        puts(\textcolor{stringliteral}{"{}Uno!"{}});}
\DoxyCodeLine{        \textcolor{keywordflow}{break};}
\DoxyCodeLine{    \textcolor{keywordflow}{default}: }
\DoxyCodeLine{        puts(\textcolor{stringliteral}{"{}Fuera de rango"{}});}
\DoxyCodeLine{\}}

\end{DoxyCode}


\DoxyHorRuler{0}


{\bfseries{Ejemplo 2\+: Múltiples Etiquetas (Fall-\/\+Through Controlado)}} ~\newline
 
\begin{DoxyCode}{0}
\DoxyCodeLine{\textcolor{keywordflow}{switch} (nota) \{}
\DoxyCodeLine{    \textcolor{keywordflow}{case} \textcolor{charliteral}{'A'}: }
\DoxyCodeLine{    \textcolor{keywordflow}{case} \textcolor{charliteral}{'B'}: }
\DoxyCodeLine{    \textcolor{keywordflow}{case} \textcolor{charliteral}{'C'}: }
\DoxyCodeLine{        puts(\textcolor{stringliteral}{"{}Aprobado"{}});}
\DoxyCodeLine{        \textcolor{keywordflow}{break};}
\DoxyCodeLine{    \textcolor{keywordflow}{case} \textcolor{charliteral}{'D'}: }
\DoxyCodeLine{    \textcolor{keywordflow}{case} \textcolor{charliteral}{'F'}: }
\DoxyCodeLine{        puts(\textcolor{stringliteral}{"{}Reprobado"{}});}
\DoxyCodeLine{        \textcolor{keywordflow}{break};}
\DoxyCodeLine{    \textcolor{keywordflow}{default}: }
\DoxyCodeLine{        puts(\textcolor{stringliteral}{"{}Nota inválida"{}});}
\DoxyCodeLine{\}}

\end{DoxyCode}


\DoxyHorRuler{0}


{\bfseries{Comparación if-\/else vs. switch}} ~\newline
 \tabulinesep=1mm
\begin{longtabu}spread 0pt [c]{*{3}{|X[-1]}|}
\hline
\PBS\centering \cellcolor{\tableheadbgcolor}\textbf{ Característica   }&\PBS\centering \cellcolor{\tableheadbgcolor}\textbf{ {\ttfamily if-\/else}   }&\PBS\centering \cellcolor{\tableheadbgcolor}\textbf{ {\ttfamily switch}    }\\\cline{1-3}
\endfirsthead
\hline
\endfoot
\hline
\PBS\centering \cellcolor{\tableheadbgcolor}\textbf{ Característica   }&\PBS\centering \cellcolor{\tableheadbgcolor}\textbf{ {\ttfamily if-\/else}   }&\PBS\centering \cellcolor{\tableheadbgcolor}\textbf{ {\ttfamily switch}    }\\\cline{1-3}
\endhead
Tipo de condiciones   &Expresiones booleanas (ej\+: {\ttfamily \texorpdfstring{$>$}{>}=5})   &Valores discretos (ej\+: {\ttfamily 5})    \\\cline{1-3}
Legibilidad   &Mejor para rangos   &Ideal para opciones fijas    \\\cline{1-3}
Eficiencia   &Puede ser menos eficiente   &Optimizado para múltiples casos    \\\cline{1-3}
Uso de {\ttfamily break}   &No aplica   &Crítico para evitar fall-\/through   \\\cline{1-3}
\end{longtabu}


\DoxyHorRuler{0}


{\bfseries{Precaución\+: Omisión de {\ttfamily break}}} ~\newline
 Si se omite {\ttfamily break}, el programa ejecutará {\bfseries{todos los casos siguientes}} hasta encontrar un {\ttfamily break} o el final del {\ttfamily switch}. Esto puede causar comportamientos inesperados. Ejemplo\+: 
\begin{DoxyCode}{0}
\DoxyCodeLine{\textcolor{keywordflow}{switch} (tipo\_vehiculo) \{}
\DoxyCodeLine{    \textcolor{keywordflow}{case} 1: }
\DoxyCodeLine{        printf(\textcolor{stringliteral}{"{}Turismo\(\backslash\)n"{}});}
\DoxyCodeLine{        \textcolor{comment}{// Sin break: ¡Ejecutará el siguiente caso!}}
\DoxyCodeLine{    \textcolor{keywordflow}{case} 2: }
\DoxyCodeLine{        printf(\textcolor{stringliteral}{"{}Autobús\(\backslash\)n"{}});}
\DoxyCodeLine{        \textcolor{keywordflow}{break};}
\DoxyCodeLine{\}}
\DoxyCodeLine{\textcolor{comment}{// Si tipo\_vehiculo es 1, imprimirá "{}Turismo"{} y "{}Autobús"{}.}}

\end{DoxyCode}


\DoxyHorRuler{0}


{\bfseries{Evaluación en Cortocircuito}} ~\newline
 En expresiones lógicas con {\ttfamily \&\&} y {\ttfamily $\vert$$\vert$}, C evalúa solo hasta donde sea necesario. Ejemplo\+: 
\begin{DoxyCode}{0}
\DoxyCodeLine{\textcolor{keywordflow}{if} (a != 0 \&\& b / a > 2) \{ }
\DoxyCodeLine{    \textcolor{comment}{// Si a es 0, evita la división (evita error)}}
\DoxyCodeLine{\}}

\end{DoxyCode}
 {\bfseries{Explicación}}\+: ~\newline

\begin{DoxyItemize}
\item En {\ttfamily \&\&}\+: Si el primer operando es falso, el segundo no se evalúa. ~\newline

\item En {\ttfamily $\vert$$\vert$}\+: Si el primer operando es verdadero, el segundo no se evalúa. ~\newline

\end{DoxyItemize}

\DoxyHorRuler{0}


{\bfseries{Usos Prácticos}} ~\newline

\begin{DoxyEnumerate}
\item {\bfseries{Menús Interactivos}}\+: ~\newline
 
\begin{DoxyCode}{0}
\DoxyCodeLine{\textcolor{keywordflow}{switch} (opcion\_menu) \{}
\DoxyCodeLine{    \textcolor{keywordflow}{case} 1: abrir\_archivo(); \textcolor{keywordflow}{break};}
\DoxyCodeLine{    \textcolor{keywordflow}{case} 2: guardar\_archivo(); \textcolor{keywordflow}{break};}
\DoxyCodeLine{    \textcolor{keywordflow}{default}: printf(\textcolor{stringliteral}{"{}Opción no válida\(\backslash\)n"{}});}
\DoxyCodeLine{\}}

\end{DoxyCode}

\item {\bfseries{Autenticación de Usuarios}}\+: ~\newline
 
\begin{DoxyCode}{0}
\DoxyCodeLine{\textcolor{keywordflow}{switch} (nivel\_acceso) \{}
\DoxyCodeLine{    \textcolor{keywordflow}{case} 1: printf(\textcolor{stringliteral}{"{}Administrador\(\backslash\)n"{}}); \textcolor{keywordflow}{break};}
\DoxyCodeLine{    \textcolor{keywordflow}{case} 2: printf(\textcolor{stringliteral}{"{}Usuario\(\backslash\)n"{}}); \textcolor{keywordflow}{break};}
\DoxyCodeLine{    \textcolor{keywordflow}{default}: printf(\textcolor{stringliteral}{"{}Invitado\(\backslash\)n"{}});}
\DoxyCodeLine{\}}

\end{DoxyCode}

\end{DoxyEnumerate}

\DoxyHorRuler{0}


{\bfseries{Conclusión}} ~\newline
 El {\ttfamily switch} es una herramienta poderosa para decisiones basadas en valores discretos. Su correcto uso depende de\+: ~\newline

\begin{DoxyItemize}
\item Elegir el tipo correcto para el selector. ~\newline

\item Incluir {\ttfamily break} donde sea necesario. ~\newline

\item Manejar casos inesperados con {\ttfamily default}. ~\newline

\end{DoxyItemize}

{\bfseries{Ejemplo Adicional -\/ Conversión de Grados}} ~\newline
 
\begin{DoxyCode}{0}
\DoxyCodeLine{\textcolor{keywordflow}{switch} (escala) \{}
\DoxyCodeLine{    \textcolor{keywordflow}{case} \textcolor{charliteral}{'C'}: }
\DoxyCodeLine{        printf(\textcolor{stringliteral}{"{}Convertir a Fahrenheit\(\backslash\)n"{}});}
\DoxyCodeLine{        \textcolor{keywordflow}{break};}
\DoxyCodeLine{    \textcolor{keywordflow}{case} \textcolor{charliteral}{'F'}: }
\DoxyCodeLine{        printf(\textcolor{stringliteral}{"{}Convertir a Celsius\(\backslash\)n"{}});}
\DoxyCodeLine{        \textcolor{keywordflow}{break};}
\DoxyCodeLine{    \textcolor{keywordflow}{default}: }
\DoxyCodeLine{        printf(\textcolor{stringliteral}{"{}Escala no soportada\(\backslash\)n"{}});}
\DoxyCodeLine{\}}

\end{DoxyCode}


\DoxyHorRuler{0}


{\bfseries{Referencia}}\+: Apuntes de páginas 132-\/137 -\/ Libro \char`\"{}\+C, C++, Java y UML\char`\"{} (Luis Joyanes). 